\documentclass[report,14pt,openright,oneside,a4paper,brazil]{abntex2}
\usepackage[utf8]{inputenc}
\usepackage{graphicx}
\usepackage{subfig}
\usepackage{lmodern}
\usepackage{float}
\usepackage[brazil]{babel}
\usepackage{booktabs}
\usepackage{longtable}
\usepackage{setspace}
\usepackage{parskip}
\usepackage{leading}
\usepackage{amsmath,esint}
\usepackage{physics}
\usepackage{indentfirst}

%Configuring the code environment

\newcommand{\abrir}[2]{\int{#1}\ket{#2}\dd{#2}}
\newcommand{\qfourier}[1]{\frac{1}{\sqrt{2\pi\hbar}}\int{#1} e^{-i \frac{p}{\hbar} x} \dd{x}}
\newcommand{\aqfourier}[1]{\frac{1}{\sqrt{2\pi\hbar}}\int{#1} e^{i \frac{p}{\hbar} x} \dd{p}}

\renewcommand{\ABNTEXchapterfont}{\fontfamily{cmr}\fontseries{b}\selectfont}
%%\renewcommand{\ABNTEXchapterfontsize}{\LARGE}

\titulo{Título}
\autor{
    \makebox[.6\textwidth]{
    	Autor A \hfill 
        RA:	00000000}\\
	\makebox[.6\textwidth]{
    	Autor B \hfill
        RA: 11111111}\\
	\makebox[.6\textwidth]{
    	Autor C \hfill 
        RA: 22222222}\\
	\makebox[.6\textwidth]{
    	Autor D \hfill 
        RA: 33333333}}
\instituicao{Universidade Federal do ABC}
\local{Santo André}
\tipotrabalho{Relatório}
\orientador{Professor\\}
\date{Data}

\renewcommand{\imprimircapa}{%
\begin{capa}%
	\begin{figure}[ht]
		\centering
		\includegraphics[scale=0.6]{logo.jpg}
		\label{fig:logo}
	\end{figure}
	\begin{center}
		\textbf{\large Disciplina - BC1419}
		\vfill
    	{\LARGE\textbf{\imprimirtitulo}}
    	\vfill
    	Docente: \imprimirorientador
    	\vspace*{1cm}
		\imprimirautor
		\vfill
    	\imprimirlocal \\
    	\imprimirdata
	\end{center}
\end{capa}
}

\setlength{\parindent}{1.3cm}
\setlength{\parskip}{0.2cm}

\begin{document}
\imprimircapa
\tableofcontents

\chapter*{Resumo}

\pagenumbering{arabic}

\chapter{Introdução}

\section{Transformada normalizada de Fourier}

\begin{equation}
    \bra{x}\ket{p}=\frac{1}{\sqrt{2\pi\hbar}}e^{i\frac{p}{\hbar}x}
\end{equation}

\begin{equation}
    \bra{p}\ket{x}=\frac{1}{\sqrt{2\pi\hbar}}e^{-i\frac{p}{\hbar}x}
\end{equation}

\begin{equation}
    \ket{\psi}=\abrir{\psi(x)}{x}=\abrir{\hat\psi(p)}{p}
\end{equation}

\begin{equation}
    \psi(x)=\bra{x}\ket{\psi}=\aqfourier{\hat \psi (p)}
\end{equation}

\begin{equation}
    \hat{\psi}(p)=\bra{p}\ket{\psi}=\qfourier{\psi(x)}
\end{equation}

\section{Propriedades}

\subsection{Transformada da derivada com relação a x}

Seja
\begin{equation}
    D_x\ket{\psi}=\ket{\dv{\psi}{x}}= \abrir{\dv{\psi}{x} (x)}{x}
\end{equation}
então
\begin{equation}
    \matrixel{p}{D_x}{\psi}= \qfourier{\dv{\psi}{x} (x)}
\end{equation}
\begin{equation}
    \matrixel{p}{D_x}{\psi}= \frac{1}{\sqrt{2\pi \hbar}}\eval{\psi(x)e^{-i\frac{p}{\hbar}x}}_{-\infty}^{+\infty}+ \frac{ip}{\hbar} \qfourier{\psi(x)}
\end{equation}
\begin{equation}
    \matrixel{p}{D_x}{\psi}= \frac{ip}{\hbar} \hat{\psi}(p)=\frac{i}{\hbar} \braket{p}{p\psi}
\end{equation}

\subsection{Transformada inversa da derivada com relação a p}

Seja
\begin{equation}
    D_p\ket{\psi}=\ket{\dv{\psi}{p}}= \abrir{\dv{\hat{\psi}}{p} (p)}{p}
\end{equation}
então
\begin{equation}
    \matrixel{x}{D_p}{\psi}= \aqfourier{\dv{\hat\psi}{p} (p)}
\end{equation}
\begin{equation}
    \matrixel{x}{D_p}{\psi}= \frac{1}{\sqrt{2\pi \hbar}}\eval{\hat\psi(p)e^{i\frac{p}{\hbar}x}}_{-\infty}^{+\infty}- \frac{ix}{\hbar} \aqfourier{\hat\psi(p)}
\end{equation}
\begin{equation}
    \matrixel{x}{D_p}{\psi}= \frac{-ix}{\hbar} \psi(x)=\frac{-i}{\hbar} \braket{x}{x\psi}
\end{equation}

\chapter{Transformada de funções}

\section{Constante}
$$\ket{f}=\abrir{A}{x}$$

\section{Gaussiana}
Seja a gaussiana
\begin{equation}
    \ket{g}=\frac{1}{\sqrt{2\pi\sigma^2}} \abrir{e^{-\frac{(x-\mu)^2}{2\sigma^2}}}{x}
\end{equation}
então
\begin{equation}
    D_x\ket{g}=\frac{1}{\sqrt{2\pi\sigma^2}} \abrir{-\frac{x-\mu}{\sigma^2} e^{-\frac{(x-\mu)^2}{2\sigma^2}}}{x}
\end{equation}
\begin{equation}
    D_x\ket{g}=\frac{1}{\sqrt{2\pi\sigma^2}} \abrir{-\frac{x}{\sigma^2} e^{-\frac{(x-\mu)^2}{2\sigma^2}}}{x}+\frac{1}{\sqrt{2\pi\sigma^2}} \abrir{\frac{\mu}{\sigma^2} e^{-\frac{(x-\mu)^2}{2\sigma^2}}}{x}
\end{equation}
\begin{equation}
    D_x\ket{g}=-\frac{1}{\sigma^2}\ket{xg}+\frac{\mu}{\sigma^2}\ket{g}
\end{equation}
\begin{equation}
    \frac{i}{\hbar}\ket{pg}=-\frac{i\hbar}{\sigma^2}D_p\ket{g}+\frac{\mu}{\sigma^2}\ket{g}
\end{equation}
\begin{equation}
    D_p\ket{g}=-\frac{\sigma^2}{\hbar^2} \ket{pg}-\frac{i\mu}{\hbar} \ket{g}
\end{equation}
\begin{equation}
    \bra{p}D_p\ket{g}=-\frac{\sigma^2}{\hbar^2} \bra{p}\ket{pg}-\frac{i\mu}{\hbar} \bra{p}\ket{g}
\end{equation}

\begin{equation}
    \dv{\hat g}{p} = -\frac{\sigma^2p}{\hbar^2} \hat{g}-\frac{i\mu}{\hbar} \hat g
\end{equation}
\begin{equation}
    \int_{\hat{g}(0)}^{\hat{g}(p)} \frac{1}{\hat g} \dd{\hat g} = \int_0^p -\frac{\sigma^2p}{\hbar^2} -\frac{i\mu}{\hbar} \dd{p} 
\end{equation}
\begin{equation}
    \ln {\hat g(p)} = -\frac{\sigma^2p^2}{2\hbar^2} -\frac{i\mu p}{\hbar} 
\end{equation}
\begin{equation}
    \hat g(p) = e^{-\frac{\sigma^2p^2}{2\hbar^2} -\frac{i\mu p}{\hbar}} 
\end{equation}

\chapter{Metodologia}

\section{Lista de materiais}

\begin{itemize}
    \item primeiro
\end{itemize}

\section{Montagem experimental}

\section{Procedimento experimental}

\chapter{Resultados e análise de dados}

\chapter{Conclusão}

\newpage
\appendix

\newpage

\addcontentsline{toc}{chapter}{\bibname}
\bibliographystyle{abntex2-num}
\bibliography{refs.bib}
\nocite{*}

\chapter{Demonstrações}

\chapter{Propagação de incertezas}

\begin{table}[H]
    \centering
    \begin{tabular}{|c|c|c|}
         \hline
         $i$ & a & b \\
         \hline
         1 & A & B \\
         2 & C & D \\
         \hline
    \end{tabular}
    \caption{Exemplo de tabela.}
    \label{tab:ex}
\end{table}

\begin{figure}[H]
    \centering
    \includegraphics[scale=0.8]{logo.jpg}
    \caption{Exemplo de imagem.}
    \label{fig:ex}
\end{figure}

\end{document}